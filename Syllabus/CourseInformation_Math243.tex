\documentclass[10pt]{article}
\usepackage{fancyhdr,amsmath,hyperref,geometry}

%%%%%%%%%%%%%%%%%%%%%%%%%%%%%%%%%%%%%% Spacing
 \textheight = 10 in
 \topmargin = 0 in
 \textwidth = 7 in
 \hoffset = -.5 in
 \voffset = -.75 in
%%%%%%%%%%%%%%%%%%%%%%%%%%%%%%%%%%%%%% New Commands
\newcommand{\Z}{\mathbb{Z}}



%%%%%%%%%%%%%%%%%%%%%%%%%%%%%%%%%%%%%% Header and Footer
\pagestyle{fancy}
\fancyhf{}
\lhead{Fall 2020 } %Term  
\chead{Math 243: Stat Learning} %Course Title and Number
\rhead{Course ID: 10301} %Course ID
%%%%%%%%%%%%%%%%%%%%%%%%%%%%%%%%%%%%%%


\begin{document}

%%%%%%%%%%%%%%%%%%%%%%%%%%%%%%%%%%%%%% Contact Information
\begin{tabular}{llll}
{\bf Instructor: } & Jonathan ``Nate'' Wells   & {\bf Email:} & \href{mailto:wellsj@reed.edu}{wellsj@reed.edu} \\ 

{\bf Classroom: } & Library 389  &{\bf Office: }   & Library 392  \\
{\bf In-person Office Hours:}  & M 3-4pm, W 10-11am, F 10-11am  &   &  \\
{\bf Virtual Office Hours:}  & T 2-3pm  &   &  \\


{\bf Zoom Link:} & \href{https://zoom.us/my/wellsj392}{https://zoom.us/my/wellsj392} & & 


\end{tabular}
%%%%%%%%%%%%%%%%%%%%%%%%%%%%%%%%%%%%%%


%%%%%%%%%%%%%%%%%%%%%%%%%%%%%%%%%%%%%% Course Description, Pre-requisites, Textbook, Technology, Required Materials, Website

\vspace{0.2in}\noindent
{\bf Course Description:} This course is an overview of modern approaches to analyzing large and complex data sets that arise in a variety of fields from biology to marketing to astrophysics. The most important modeling and predictive techniques will be covered, including regression, classification, clustering, resampling, and tree-based methods. There will be several projects throughout the course, which will require significant programming in R.

\vspace{0.1in}\noindent
{\bf Prerequisites:}  MATH141, or Instructor Consent.

\vspace{0.1in}\noindent
{\bf Distribution Requirements:} This course can be used towards your Group III, “Natural, Mathematical, and Psychological Science,” requirement. It accomplishes the following learning goals for the group:
\small
\begin{quotation}
 Use and evaluate quantitative data or modeling, or use logical/mathematical reasoning to evaluate, test or prove statements; Given a problem or question, formulate a hypothesis or conjecture, and design an experiment, collect data or use mathematical reasoning to test or validate it; Collect, interpret and analyze data
\end{quotation}
\normalsize
	This course does not satisfy the ``primary data collection and analysis'' requirement.


\vspace{0.1in}\noindent
{\bf Textbooks:} 
\small
\begin{itemize}
\item (Primary) \textit{An Introduction to Statistical Learning}, 2nd Edition by James, Witten, Hastie, and Tibshirani. Free online access to the textbook is available via SpringerLink
\item (Secondary) \textit{Applied Predictive Modeling}, 1st Edition by Kuhn and Johnson. We will use select content from this text to supplement \textit{ISL}. Free online access to the textbook is available at SpringerLink
\end{itemize}  
\normalsize
 


\vspace{0.1in}\noindent
{\bf Course Resources:} 
The following web-based resources will be used for communicating class information:
\small
\begin{itemize}
\item Slack \url{reedmath391fall2021.slack.com} (\textit{announcements, discussions, direct messaging})
\item Course Website \url{https://Reed-Stat-Learning-Fall-2021.github.io} (\textit{documents, a daily schedule, assignments}).
\item GitHub Classroom \url{https://classroom.github.com/classrooms/88862064-reed-stat-learning-fall-2021-classroom} (\textit{homework submission})
\end{itemize}
\normalsize

\vspace{0.1in}\noindent
{\bf Technology:} You are encouraged to bring a personal computer to class each day for notetaking and live coding. Access to a computer with webbrowser will be required for homework completion and submission. Computing \& Information Services offers programs for long-term laptop loan: \url{https://www.reed.edu/cis/facilities/student-technology-equipment-program.html}

 \vspace{0.1in}\noindent We will make very frequent use of the R programming language to create statistical models, run simulations, and implement stat learning algorithms. All homework will be completed using the RStudio IDE. R and RStudio are free to use, and can either be installed locally on your computer, or can be accessed using the Reed RStudio Server: \url{https://rstudio.reed.edu/}

 \vspace{0.1in}\noindent Throughout the term, we will use GitHub to manage and submit assignments. GitHub is a hosting service to house Git-based projects online, and is designed to assist with version control and collaboration on big projects. \url{https://github.com/}
 
 \vspace{0.1in}\noindent
{\bf Communication:} If you would like to contact me, I can most easily be reached via Slack message weekdays between 8am and 6pm. While I try to answer messages as soon as possible, in some cases, I may not be able to respond until the following school day. If you'd prefer to talk live, send me a message and we can schedule a time to chat on zoom. 
 
 
%%%%%%%%%%%%%%%%%%%%%%%%%%%%%%%%%%%%%% Course Outcomes

\vspace{0.1in}\noindent
{\bf Course Outcomes:}
By the end of the course, a student should be able to:
\small
\begin{enumerate}
\item Articulate and compare the different philosophical approaches to prediction, statistical inference, classification, and clustering.
\item Create valid statistical models, perform data analysis using software, and communicate results in non-technical language using reproducible methods in order to answer a particular research question.
\item Implement simulation and randomization algorithms in order to demonstrate and assess properties of statistical models.
\item Assess and compare the performance of a variety of statistical models, and select appropriate models according to suitable criteria.
\item Apply statistical learning techniques to real-world data and problems.
\item Justify and describe properties of particular statistical learning methods by appealing to mathematical theory.
\end{enumerate}
\normalsize
 
\end{document}






