% Options for packages loaded elsewhere
\PassOptionsToPackage{unicode}{hyperref}
\PassOptionsToPackage{hyphens}{url}
%
\documentclass[
  9pt,
  ignorenonframetext,
]{beamer}
\title{Principal Component Regression}
\author{Nate Wells}
\date{December 3rd, 2021}
\institute{Math 243: Stat Learning}

\usepackage{pgfpages}
\setbeamertemplate{caption}[numbered]
\setbeamertemplate{caption label separator}{: }
\setbeamercolor{caption name}{fg=normal text.fg}
\beamertemplatenavigationsymbolsempty
% Prevent slide breaks in the middle of a paragraph
\widowpenalties 1 10000
\raggedbottom
\setbeamertemplate{part page}{
  \centering
  \begin{beamercolorbox}[sep=16pt,center]{part title}
    \usebeamerfont{part title}\insertpart\par
  \end{beamercolorbox}
}
\setbeamertemplate{section page}{
  \centering
  \begin{beamercolorbox}[sep=12pt,center]{part title}
    \usebeamerfont{section title}\insertsection\par
  \end{beamercolorbox}
}
\setbeamertemplate{subsection page}{
  \centering
  \begin{beamercolorbox}[sep=8pt,center]{part title}
    \usebeamerfont{subsection title}\insertsubsection\par
  \end{beamercolorbox}
}
\AtBeginPart{
  \frame{\partpage}
}
\AtBeginSection{
  \ifbibliography
  \else
    \frame{\sectionpage}
  \fi
}
\AtBeginSubsection{
  \frame{\subsectionpage}
}
\usepackage{amsmath,amssymb}
\usepackage{lmodern}
\usepackage{iftex}
\ifPDFTeX
  \usepackage[T1]{fontenc}
  \usepackage[utf8]{inputenc}
  \usepackage{textcomp} % provide euro and other symbols
\else % if luatex or xetex
  \usepackage{unicode-math}
  \defaultfontfeatures{Scale=MatchLowercase}
  \defaultfontfeatures[\rmfamily]{Ligatures=TeX,Scale=1}
\fi
% Use upquote if available, for straight quotes in verbatim environments
\IfFileExists{upquote.sty}{\usepackage{upquote}}{}
\IfFileExists{microtype.sty}{% use microtype if available
  \usepackage[]{microtype}
  \UseMicrotypeSet[protrusion]{basicmath} % disable protrusion for tt fonts
}{}
\makeatletter
\@ifundefined{KOMAClassName}{% if non-KOMA class
  \IfFileExists{parskip.sty}{%
    \usepackage{parskip}
  }{% else
    \setlength{\parindent}{0pt}
    \setlength{\parskip}{6pt plus 2pt minus 1pt}}
}{% if KOMA class
  \KOMAoptions{parskip=half}}
\makeatother
\usepackage{xcolor}
\IfFileExists{xurl.sty}{\usepackage{xurl}}{} % add URL line breaks if available
\IfFileExists{bookmark.sty}{\usepackage{bookmark}}{\usepackage{hyperref}}
\hypersetup{
  pdftitle={Principal Component Regression},
  pdfauthor={Nate Wells},
  hidelinks,
  pdfcreator={LaTeX via pandoc}}
\urlstyle{same} % disable monospaced font for URLs
\newif\ifbibliography
\usepackage{color}
\usepackage{fancyvrb}
\newcommand{\VerbBar}{|}
\newcommand{\VERB}{\Verb[commandchars=\\\{\}]}
\DefineVerbatimEnvironment{Highlighting}{Verbatim}{commandchars=\\\{\}}
% Add ',fontsize=\small' for more characters per line
\usepackage{framed}
\definecolor{shadecolor}{RGB}{248,248,248}
\newenvironment{Shaded}{\begin{snugshade}}{\end{snugshade}}
\newcommand{\AlertTok}[1]{\textcolor[rgb]{0.94,0.16,0.16}{#1}}
\newcommand{\AnnotationTok}[1]{\textcolor[rgb]{0.56,0.35,0.01}{\textbf{\textit{#1}}}}
\newcommand{\AttributeTok}[1]{\textcolor[rgb]{0.77,0.63,0.00}{#1}}
\newcommand{\BaseNTok}[1]{\textcolor[rgb]{0.00,0.00,0.81}{#1}}
\newcommand{\BuiltInTok}[1]{#1}
\newcommand{\CharTok}[1]{\textcolor[rgb]{0.31,0.60,0.02}{#1}}
\newcommand{\CommentTok}[1]{\textcolor[rgb]{0.56,0.35,0.01}{\textit{#1}}}
\newcommand{\CommentVarTok}[1]{\textcolor[rgb]{0.56,0.35,0.01}{\textbf{\textit{#1}}}}
\newcommand{\ConstantTok}[1]{\textcolor[rgb]{0.00,0.00,0.00}{#1}}
\newcommand{\ControlFlowTok}[1]{\textcolor[rgb]{0.13,0.29,0.53}{\textbf{#1}}}
\newcommand{\DataTypeTok}[1]{\textcolor[rgb]{0.13,0.29,0.53}{#1}}
\newcommand{\DecValTok}[1]{\textcolor[rgb]{0.00,0.00,0.81}{#1}}
\newcommand{\DocumentationTok}[1]{\textcolor[rgb]{0.56,0.35,0.01}{\textbf{\textit{#1}}}}
\newcommand{\ErrorTok}[1]{\textcolor[rgb]{0.64,0.00,0.00}{\textbf{#1}}}
\newcommand{\ExtensionTok}[1]{#1}
\newcommand{\FloatTok}[1]{\textcolor[rgb]{0.00,0.00,0.81}{#1}}
\newcommand{\FunctionTok}[1]{\textcolor[rgb]{0.00,0.00,0.00}{#1}}
\newcommand{\ImportTok}[1]{#1}
\newcommand{\InformationTok}[1]{\textcolor[rgb]{0.56,0.35,0.01}{\textbf{\textit{#1}}}}
\newcommand{\KeywordTok}[1]{\textcolor[rgb]{0.13,0.29,0.53}{\textbf{#1}}}
\newcommand{\NormalTok}[1]{#1}
\newcommand{\OperatorTok}[1]{\textcolor[rgb]{0.81,0.36,0.00}{\textbf{#1}}}
\newcommand{\OtherTok}[1]{\textcolor[rgb]{0.56,0.35,0.01}{#1}}
\newcommand{\PreprocessorTok}[1]{\textcolor[rgb]{0.56,0.35,0.01}{\textit{#1}}}
\newcommand{\RegionMarkerTok}[1]{#1}
\newcommand{\SpecialCharTok}[1]{\textcolor[rgb]{0.00,0.00,0.00}{#1}}
\newcommand{\SpecialStringTok}[1]{\textcolor[rgb]{0.31,0.60,0.02}{#1}}
\newcommand{\StringTok}[1]{\textcolor[rgb]{0.31,0.60,0.02}{#1}}
\newcommand{\VariableTok}[1]{\textcolor[rgb]{0.00,0.00,0.00}{#1}}
\newcommand{\VerbatimStringTok}[1]{\textcolor[rgb]{0.31,0.60,0.02}{#1}}
\newcommand{\WarningTok}[1]{\textcolor[rgb]{0.56,0.35,0.01}{\textbf{\textit{#1}}}}
\usepackage{graphicx}
\makeatletter
\def\maxwidth{\ifdim\Gin@nat@width>\linewidth\linewidth\else\Gin@nat@width\fi}
\def\maxheight{\ifdim\Gin@nat@height>\textheight\textheight\else\Gin@nat@height\fi}
\makeatother
% Scale images if necessary, so that they will not overflow the page
% margins by default, and it is still possible to overwrite the defaults
% using explicit options in \includegraphics[width, height, ...]{}
\setkeys{Gin}{width=\maxwidth,height=\maxheight,keepaspectratio}
% Set default figure placement to htbp
\makeatletter
\def\fps@figure{htbp}
\makeatother
\setlength{\emergencystretch}{3em} % prevent overfull lines
\providecommand{\tightlist}{%
  \setlength{\itemsep}{0pt}\setlength{\parskip}{0pt}}
\setcounter{secnumdepth}{-\maxdimen} % remove section numbering
\usetheme{Madrid}
\useoutertheme{miniframes} % Alternatively: miniframes, infolines, split
\useinnertheme{circles}

\definecolor{ReedRed}{RGB}{167, 14, 22} %Reed Red (primary)
%\definecolor{UOGreen}{RGB}{3, 105, 54} % UO Green (primary)

\usecolortheme[named=ReedRed]{structure}
%\usecolortheme[named=Mahogany]{structure} % Sample dvipsnames color


%\usepackage[urw-garamond]{mathdesign}
%\usepackage[T1]{fontenc}

\makeatletter
\g@addto@macro\normalsize{%
    \setlength\belowdisplayskip{-0pt}
}

\newcommand{\columnsbegin}{\begin{columns}}
\newcommand{\columnsend}{\end{columns}}
\ifLuaTeX
  \usepackage{selnolig}  % disable illegal ligatures
\fi

\begin{document}
\frame{\titlepage}

\begin{frame}{Outline}
\protect\hypertarget{outline}{}
In today's class, we will\ldots{}

\begin{itemize}
\item
  Discuss Principal Component Analysis as a means of dimensionality
  reduction for regresion
\item
  Implement PCR in R
\end{itemize}
\end{frame}

\hypertarget{principal-component-regression}{%
\section{Principal Component
Regression}\label{principal-component-regression}}

\begin{frame}{Dimensionality Reduction}
\protect\hypertarget{dimensionality-reduction}{}
Suppose you collect a sample of \(n\) observations on \(p\) predictors
\(X_1, \dots, X_p\), where \(p\) is relatiely large. Suppose further
that some of the predictors are correlated with one another.

\pause

\begin{itemize}
\tightlist
\item
  Any predictive model for a response \(Y\) based on all of the
  correlated variables will underperform due to instability in parameter
  estimates.
\end{itemize}

\pause

It may be difficult to fit complex models accurately, given limited
number of observatiosn compared to predictors.

\pause

\begin{itemize}
\tightlist
\item
  If \(p\) is larger than \(n\), it may not be possible to fit certain
  models to the data (for example MLR models cannot be used)
\end{itemize}

\pause

One solution is to perform variable selection and drop some less useful
predictors.

\pause

\begin{itemize}
\tightlist
\item
  But dropping variables completely loses possible valauable
  information.
\end{itemize}

\pause

\begin{itemize}
\tightlist
\item
  Instead, we can combine variables into new ones that adequately
  describe the variance in the data, and drop those that have limited
  utility in explaining that variance.
\end{itemize}
\end{frame}

\begin{frame}[fragile]{PCA Overview}
\protect\hypertarget{pca-overview}{}
Consider the relationship between campaign ad spending and population
size for 100 cities:

\begin{center}\includegraphics[width=0.45\linewidth]{12_3_PCR_files/figure-beamer/unnamed-chunk-1-1} \end{center}

\pause
\normalsize

What are the approximate standard deviations of ad spending and
population?

\pause

\small

\begin{verbatim}
##     sd_Pop    sd_Ad
## 1 8.981994 7.418227
\end{verbatim}
\end{frame}

\begin{frame}[fragile]{PCA Overview}
\protect\hypertarget{pca-overview-1}{}
Consider the relationship between campaign ad spending and population
size for 100 cities:

\begin{center}\includegraphics[width=0.45\linewidth]{12_3_PCR_files/figure-beamer/unnamed-chunk-3-1} \end{center}

\normalsize

But how much of the variation in ad spending is just due to variation in
population?

\small
\pause

\begin{verbatim}
##        R_sq
## 1 0.8238886
\end{verbatim}
\end{frame}

\begin{frame}{PCA Overview}
\protect\hypertarget{pca-overview-2}{}
Can we find a line along which the observations vary the most?

\pause

\begin{center}\includegraphics[width=0.45\linewidth]{12_3_PCR_files/figure-beamer/unnamed-chunk-6-1} \end{center}
\end{frame}

\begin{frame}{PCA Overview}
\protect\hypertarget{pca-overview-3}{}
How much variation occurs perpendicular to this line?

\begin{center}\includegraphics[width=0.45\linewidth]{12_3_PCR_files/figure-beamer/unnamed-chunk-7-1} \end{center}
\end{frame}

\begin{frame}{PCA Definition}
\protect\hypertarget{pca-definition}{}
The \emph{first principal component} \(Z_1\) is the direction along
which there is the greatest variability in the data.

\pause

\begin{itemize}
\tightlist
\item
  That is, if we project the observations onto this line, the resulting
  projected observations would have the greatest possible variance.
\end{itemize}

\pause

\begin{itemize}
\tightlist
\item
  Projecting a point onto a line amounts to finding the location on the
  line closest to the given point.
\end{itemize}

\pause

We can express the first principal component as a linear combination of
the centered predictors \(X_i - \bar{X_i}\), where
\(\phi_{i1} \in \mathrm{R}\) and
\(\phi_{11}^2+ \dots + \phi_{p1}^2 = 1\):

\pause

\[
Z_1 = \phi_{11} (X_1 - \bar{X}_1) + \phi_{21}  (X_2 - \bar{X}_2) + \dots + \phi_{p1} (X_p - \bar{X}_p) 
\]

\pause

\begin{itemize}
\tightlist
\item
  Alternatively, we could express \(Z_1\) as an affine linear
  combination of the predictors themselves (affine meaning including a
  constant term)
\end{itemize}
\end{frame}

\begin{frame}{PCA Example}
\protect\hypertarget{pca-example}{}
The first principal component

\begin{center}\includegraphics[width=0.45\linewidth]{12_3_PCR_files/figure-beamer/unnamed-chunk-8-1} \end{center}

\[
Z_1 = 0.8 (\textrm{Pop} - 41.1 )  + 0.6 (\textrm{Ad} - 40.4)
\]
\end{frame}

\begin{frame}{PCA Example}
\protect\hypertarget{pca-example-1}{}
What is leftover?

\begin{center}\includegraphics[width=0.75\linewidth]{12_3_PCR_files/figure-beamer/unnamed-chunk-9-1} \end{center}
\end{frame}

\begin{frame}{Other Principal Components}
\protect\hypertarget{other-principal-components}{}
In general, if we have \(p\) predictors, we can compute \(p\) distinct
principal components: \(Z_1, Z_2, \dots , Z_p\).

\pause

The second principal component \(Z_2\) is a linear combination of the
centered variables that is

\begin{itemize}
\tightlist
\item
  uncorrelated with the first principal component
\item
  has the largest variance subject to this constraint.
\end{itemize}

\pause

For the case when \(p =2\), the 2nd principal component corresponds to
the line perpendicular to the line for the 1st principal component.

\pause

Generally, the \(k\)th principal component is obtained by finding a
linear combination of centered variables that is uncorrelated with all
previous principal components, and has the largest variance subject to
this constraint.
\end{frame}

\begin{frame}{PCA Example}
\protect\hypertarget{pca-example-2}{}
The second principal component

\begin{center}\includegraphics[width=0.45\linewidth]{12_3_PCR_files/figure-beamer/unnamed-chunk-10-1} \end{center}

\[
Z_2 = 0.6 (\textrm{Pop} - 41.1 )  - 0.8 (\textrm{Ad} - 40.4)
\]
\end{frame}

\begin{frame}{Principal Comoponent Regression}
\protect\hypertarget{principal-comoponent-regression}{}
The PCR approach to linear regression constructs the first \(M\)
principal components \(Z_1, \dots, Z_M\) of a data set with \(p\)
predictors (so \(M \leq p\)), and then uses these as predictors in a
linear regression model.

\pause

\begin{itemize}
\tightlist
\item
  Goal: Use a small number of predictors which explain most of the
  variability in the data set, as well as their relationship to the
  response.
\end{itemize}

\pause

In general, PCR tends to produce linear models with higher accuracy than
models fit with the original predictors.

\begin{center}\includegraphics[width=0.7\linewidth]{12_3_PCR_files/figure-beamer/unnamed-chunk-11-1} \end{center}
\end{frame}

\begin{frame}[fragile]{Principal Component Regression in R}
\protect\hypertarget{principal-component-regression-in-r}{}
We can use the \texttt{pcr} function in the \texttt{pls} library to
quickly perform PCR in R.

\pause

The \texttt{Hitters} data set from the \texttt{ISLR} package contains
\texttt{Salary} and 18 other predictors for 263 baseball players

\small

\begin{Shaded}
\begin{Highlighting}[]
\FunctionTok{set.seed}\NormalTok{(}\DecValTok{1}\NormalTok{)}
\FunctionTok{library}\NormalTok{(pls)}
\NormalTok{my\_pcr }\OtherTok{\textless{}{-}} \FunctionTok{pcr}\NormalTok{( Salary }\SpecialCharTok{\textasciitilde{}}\NormalTok{ ., }\AttributeTok{data =}\NormalTok{ Hitters, }\AttributeTok{scale =}\NormalTok{ T, }\AttributeTok{validation =} \StringTok{"CV"}\NormalTok{)}
\end{Highlighting}
\end{Shaded}

\pause

\begin{itemize}
\item
  Setting \texttt{scale\ =\ T} standardizes each predictor
\item
  Setting \texttt{validation\ =\ "CV"} causes \texttt{pcr} to compute
  the 10-fold CV error for each value of \(M\) (number of principal
  components used)
\end{itemize}
\end{frame}

\begin{frame}[fragile]{PCR Results}
\protect\hypertarget{pcr-results}{}
\tiny

\begin{Shaded}
\begin{Highlighting}[]
\FunctionTok{summary}\NormalTok{(my\_pcr)}
\end{Highlighting}
\end{Shaded}

\begin{verbatim}
## Data:    X dimension: 263 19 
##  Y dimension: 263 1
## Fit method: svdpc
## Number of components considered: 19
## 
## VALIDATION: RMSEP
## Cross-validated using 10 random segments.
##        (Intercept)  1 comps  2 comps  3 comps  4 comps  5 comps  6 comps
## CV             452    352.5    351.6    352.3    350.7    346.1    345.5
## adjCV          452    352.1    351.2    351.8    350.1    345.5    344.6
##        7 comps  8 comps  9 comps  10 comps  11 comps  12 comps  13 comps
## CV       345.4    348.5    350.4     353.2     354.5     357.5     360.3
## adjCV    344.5    347.5    349.3     351.8     353.0     355.8     358.5
##        14 comps  15 comps  16 comps  17 comps  18 comps  19 comps
## CV        352.4     354.3     345.6     346.7     346.6     349.4
## adjCV     350.2     352.3     343.6     344.5     344.3     346.9
## 
## TRAINING: % variance explained
##         1 comps  2 comps  3 comps  4 comps  5 comps  6 comps  7 comps  8 comps
## X         38.31    60.16    70.84    79.03    84.29    88.63    92.26    94.96
## Salary    40.63    41.58    42.17    43.22    44.90    46.48    46.69    46.75
##         9 comps  10 comps  11 comps  12 comps  13 comps  14 comps  15 comps
## X         96.28     97.26     97.98     98.65     99.15     99.47     99.75
## Salary    46.86     47.76     47.82     47.85     48.10     50.40     50.55
##         16 comps  17 comps  18 comps  19 comps
## X          99.89     99.97     99.99    100.00
## Salary     53.01     53.85     54.61     54.61
\end{verbatim}

\small

\pause

\begin{itemize}
\tightlist
\item
  Note: \texttt{pcr} reports RSE, so values need to be squared to get
  MSE.
\end{itemize}
\end{frame}

\begin{frame}[fragile]{Validation Plot}
\protect\hypertarget{validation-plot}{}
\small

\begin{Shaded}
\begin{Highlighting}[]
\FunctionTok{validationplot}\NormalTok{(my\_pcr, }\AttributeTok{val.type =} \StringTok{"MSEP"}\NormalTok{)}
\end{Highlighting}
\end{Shaded}

\begin{center}\includegraphics[width=0.6\linewidth]{12_3_PCR_files/figure-beamer/unnamed-chunk-15-1} \end{center}

\pause

\normalsize

\begin{itemize}
\tightlist
\item
  Note: The smallest CV error occurs at \(M = 16\) (which is close to
  the maximum number of predictors \(p = 19\).)
\end{itemize}

\pause

\begin{itemize}
\tightlist
\item
  However, a relatively low CV error is also obtained at \(M = 6\),
  suggesting fewer components are sufficient
\end{itemize}
\end{frame}

\begin{frame}{Comparative Performance}
\protect\hypertarget{comparative-performance}{}
Live coding. A .Rmd file will be available on course website after class
\end{frame}

\end{document}
